% IN THE NAME OF GOD
% LaTeX Problem Set Template by Sachin Padmanabhan edited by Ali Heydari
% I created this when I was a freshman in CS 103,
% and I continue to use it to this day.
%
% Hope you enjoy!
%
% There may be problems with this template.
% If so, feel free to contact me.
%

\documentclass[a4paper]{article}
\usepackage{amsmath}
\usepackage{amssymb}
\usepackage{amsthm}
\usepackage{amssymb}
\usepackage{mathdots}
\usepackage[pdftex]{graphicx}
\usepackage{fancyhdr}
\usepackage[margin=1in]{geometry}
\usepackage{multicol}
\usepackage{bm}
\usepackage{listings}
\PassOptionsToPackage{usenames,dvipsnames}{color}  %% Allow color names
\usepackage{pdfpages}
\usepackage{algpseudocode}
\usepackage{tikz}
\usepackage{enumitem}
\usepackage[T1]{fontenc}
\usepackage{inconsolata}
\usepackage{framed}
\usepackage{wasysym}
\usepackage[thinlines]{easytable}
\usepackage{hyperref}
\usepackage{dsfont}


\usepackage{wrapfig}
\setlength{\intextsep}{0pt}
\setlength{\columnsep}{0pt}
\usepackage{subcaption}
\usepackage{graphicx}
\graphicspath{ {images/} }

\hypersetup{
    colorlinks=true,
    linkcolor=blue,
    filecolor=magenta,
    urlcolor=blue,
}

\title{
\textsc{Iran university of science and technology} \\ [25pt] % Your university, school and/or department name(s)
Discrete mathematics\\Problem Set \#3 \\
}
\author{Ali Heydari}
\date{\today}

\lhead{Ali Heydari}
\chead{Problem Set \#3}
\rhead{\today}
\lfoot{}
\cfoot{Discrete mathematics --- Winter 2018}
\rfoot{\thepage}

\newcommand{\abs}[1]{\lvert #1 \rvert}
\newcommand{\absfit}[1]{\left\lvert #1 \right\rvert}
\newcommand{\norm}[1]{\left\lVert #1 \right\rVert}
\newcommand{\eval}[3]{\left[#1\right]_{#2}^{#3}}
\renewcommand{\(}{\left(}
\renewcommand{\)}{\right)}
\newcommand{\floor}[1]{\left\lfloor#1\right\rfloor}
\newcommand{\ceil}[1]{\left\lceil#1\right\rceil}
\newcommand{\pd}[1]{\frac{\partial}{\partial #1}}
\newcommand{\inner}[1]{\langle#1\rangle}
\newcommand{\cond}{\bigg|}
\newcommand{\rank}[1]{\mathbf{rank}(#1)}
\newcommand{\range}[1]{\mathbf{range}(#1)}
\newcommand{\nullsp}[1]{\mathbf{null}(#1)}
\newcommand{\repr}[1]{\left\langle#1\right\rangle}

\DeclareMathOperator{\Var}{Var}
\DeclareMathOperator{\tr}{tr}
\DeclareMathOperator{\Tr}{\mathbf{Tr}}
\DeclareMathOperator{\diag}{\mathbf{diag}}
\DeclareMathOperator{\dist}{\mathbf{dist}}
\DeclareMathOperator{\prob}{\mathbf{prob}}
\DeclareMathOperator{\dom}{\mathbf{dom}}
\DeclareMathOperator{\E}{\mathbf{E}}
\DeclareMathOperator{\R}{\mathbb{R}}
\DeclareMathOperator{\var}{\mathbf{var}}
\DeclareMathOperator{\quartile}{\mathbf{quartile}}
\DeclareMathOperator{\conv}{\mathbf{conv}}
\DeclareMathOperator{\VC}{VC}
\DeclareMathOperator*{\argmax}{arg\,max}
\DeclareMathOperator*{\argmin}{arg\,min}
\DeclareMathOperator{\Ber}{Bernoulli}
\DeclareMathOperator{\NP}{\mathbf{NP}}
\DeclareMathOperator{\coNP}{\mathbf{coNP}}
\DeclareMathOperator{\TIME}{\mathsf{TIME}}
\DeclareMathOperator{\polytime}{\mathbf{P}}
\DeclareMathOperator{\PH}{\mathbf{PH}}
\DeclareMathOperator{\SIZE}{\mathbf{SIZE}}
\DeclareMathOperator{\ATIME}{\mathbf{ATIME}}
\DeclareMathOperator{\SPACE}{\mathbf{SPACE}}
\DeclareMathOperator{\ASPACE}{\mathbf{ASPACE}}
\DeclareMathOperator{\NSPACE}{\mathbf{NSPACE}}
\DeclareMathOperator{\Z}{\mathbb{Z}}
\DeclareMathOperator{\N}{\mathbb{N}}
\DeclareMathOperator{\EXP}{\mathbf{EXP}}
\DeclareMathOperator{\NEXP}{\mathbf{NEXP}}
\DeclareMathOperator{\NTIME}{\mathbf{NTIME}}
\DeclareMathOperator{\DTIME}{\mathbf{DTIME}}
\DeclareMathOperator{\poly}{poly}
\DeclareMathOperator{\BPP}{\mathbf{BPP}}
\DeclareMathOperator{\ZPP}{\mathbf{ZPP}}
\DeclareMathOperator{\RP}{\mathbf{RP}}
\DeclareMathOperator{\coRP}{\mathbf{coRP}}
\DeclareMathOperator{\BPL}{\mathbf{BPL}}
\DeclareMathOperator{\IP}{\mathbf{IP}}
\DeclareMathOperator{\PSPACE}{\mathbf{PSPACE}}
\DeclareMathOperator{\NPSPACE}{\mathbf{NPSPACE}}
\DeclareMathOperator{\SAT}{\mathsf{SAT}}
\DeclareMathOperator{\NL}{\mathbf{NL}}
\DeclareMathOperator{\PCP}{\mathbf{PCP}}
\DeclareMathOperator{\PP}{\mathbf{PP}}
\DeclareMathOperator{\cost}{cost}
\let\Pr\relax
\DeclareMathOperator*{\Pr}{\mathbf{Pr}}

\definecolor{shadecolor}{gray}{0.95}

\theoremstyle{plain}
\newtheorem*{lem}{Lemma}

\theoremstyle{plain}
\newtheorem*{claim}{Claim}

\theoremstyle{definition}
\newtheorem*{answer}{Answer}

\newtheorem{theorem}{Theorem}[section]
\newtheorem*{thm}{Theorem}
\newtheorem{corollary}{Corollary}[theorem]
\newtheorem{lemma}[theorem]{Lemma}

\renewcommand{\headrulewidth}{0.4pt}
\renewcommand{\footrulewidth}{0.4pt}

\setlength{\parindent}{0pt}

\pagestyle{fancy}

\renewcommand{\thefootnote}{\fnsymbol{footnote}}

\begin{document}

\maketitle

\section{Give the predicates}
In each of the following, write a statement in first-order logic that expresses the indicated sentence. Your statement may use any first-order construct (equality, connectives, quantifiers, etc.), but you must only use the predicates, functions, and constants provided. You do not need to provide the simplest formula possible, though we'd appreciate it if you made an effort to do so.
\begin{enumerate}[label*=\roman*.,ref=\roman*]

\item Given the predicates
\begin{itemize}
\item[] \textit{Natural(x)}, which states that $x$ is an natural number
\end{itemize}
and the functions
\begin{itemize}
\item[] $x + y$, which represents the sum of $x$ and $y$, and
\item[] $x . y $, which represents the product of $x$ and $y$
\end{itemize}

write a statement in first-order logic that says "for any $n \in \mathds{N} $ , n is even if and only if $n^2$ is even."
\begin{shaded}
\begin{answer}
ANSWER HERE.
\end{answer}
\end{shaded}

\item Given the predicate
\begin{itemize}
  \item[] \textit{Person(p)}, which states that $p$ is a person;
  \item[] \textit{Kitten(k)}, which states that $k$ is a kitten; and
  \item[] \textit{HasPet(o, p)}, which states that $o$ has $p$ as a pet,
\end{itemize}
write an FOL statement that says "someone has exactly two pet kittens and no other pets."
\begin{shaded}
\begin{answer}
ANSWER HERE.
\end{answer}
\end{shaded}

\item The \textbf{\textit{axiom of pairing}} is the following statement: given any two distinct objects $x$ and $y$, there's a set containing $x$ and $y$ and nothing else. Given the predicates
\begin{itemize}
  \item[] $x \in y$, which states that $x$ is an element of $y$, and
  \item[] \textit{Set(S)}, which states that $S$ is a set,
\end{itemize}
    write a statement in first-order logic that expresses the axiom of pairing.
\begin{shaded}
\begin{answer}
ANSWER HERE.
\end{answer}
\end{shaded}

\item Given the predicates
\begin{itemize}
\item[] $x \in  y$, which states that $x$ is an element of $y$, and
\item[] \textit{Set(S)}, which states that $S$ is a set,
\end{itemize}
write a statement in first-order logic that says "every set has a power set."
\begin{shaded}
\begin{answer}
ANSWER HERE.
\end{answer}
\end{shaded}

\item Given the predicates
\begin{itemize}
\item[] \textit{Lady(x)}, which states that $x$ is a lady;
\item[] \textit{Glitters(x)}, which states that $x$ glitters;
\item[] \textit{SureIsGold(x, v)}, which states that $x$ is sure that $y$ is gold;
\item[] \textit{Buying(x, v)}, which states that $x$ buys $y$; and
\item[] \textit{StuirwayToHeaven(A)}, which states that $x$ is a Stairway to Heaven;
\end{itemize}
write a statement in first-order logic that says "there's a lady who's sure all that glitters is gold, and she's buying a Stairway to Heaven."
\begin{shaded}
\begin{answer}
ANSWER HERE.
\end{answer}
\end{shaded}

\end{enumerate}

\section{Consistent vocabulary}
Represent the following sentences in first-order logic, using a consistent vocabulary (which you must define):
\begin{enumerate}[label*=\alph*.,ref=\alph*]
\item Some students took French in spring 2001.
\begin{shaded}
\begin{answer}
\begin{equation*}
  \exists x , \exists y \quad : \quad Student(x) \wedge French(y) \wedge TakeInSpring2001(x,y)
\end{equation*}
\end{answer}
\end{shaded}

\item Every student who takes French passes it.
\begin{shaded}
\begin{answer}
\begin{equation*}
  \forall x , \forall y \quad : \quad \left( Student(x) \wedge French(y) \wedge Take(x,y) \right) \quad \Rightarrow \quad Pass(x,y)
\end{equation*}
\end{answer}
\end{shaded}

\item Only one student took Greek in spring 2001.
\begin{shaded}
\begin{answer}
\begin{align*}
    \exists x , \exists y , \forall z &: \\
 & Student(x) \wedge Greek(y) \wedge TakeInSpring2001 \wedge \underbrace{(Student(z) \wedge TakeInSpring2001(z,y))}_{x = z}
\end{align*}
\end{answer}
\end{shaded}

\item The best score in Greek is always higher than the best score in French.
\begin{shaded}
\begin{answer}
\begin{equation*}
  \forall s \exists x \forall y Score(x,Greek, s) > Score(y,French, s).
\end{equation*}
\end{answer}
\end{shaded}

\item Every person who buys a policy is smart.
\begin{shaded}
\begin{answer}
\begin{equation*}
  \forall x \quad : \quad Person(x) \wedge (\exists y : Policy(y) \wedge Buys(x,y) ) \quad \Rightarrow \quad Smart(x)
\end{equation*}
\end{answer}
\end{shaded}

\item No person buys an expensive policy.
\begin{shaded}
\begin{answer}
\begin{align*}
  \forall x, y & : \\
   & \quad Person(x) \wedge Policy(y)  \wedge Expensive(y) \quad \Rightarrow \quad  \neg Buys(x, y)
\end{align*}
\end{answer}
\end{shaded}

\item There is an agent who sells policies only to people who are not insured.
\begin{shaded}
\begin{answer}
\begin{equation*}
 \exists x Agent(x) \wedge \forall y, z Policy(y) \wedge Sells(x, y, z) \quad \Rightarrow \quad (Person(z)\wedge \neg Insured(z)).
\end{equation*}
\end{answer}
\end{shaded}

\item There is a barber who shaves all men in town who do not shave themselves.
\begin{shaded}
\begin{answer}
\begin{equation*}
  \exists x \quad Barber(x) \wedge \forall y \quad Man(y) \wedge \neg Shaves(y, y) \quad \Rightarrow \quad Shaves(x, y).
\end{equation*}
\end{answer}
\end{shaded}

\item A person born in the UK. each of whose parents is a UK citizen or a UK resident, is a UK citizen by birth.
\begin{shaded}
\begin{answer}
\begin{align*}
  \forall x \, Person(x) \wedge Born(x,UK) \wedge (\forall y \, Parent(y, x)  &\Rightarrow  ((\exists r \, Citizen(y,UK, r)) \vee Resident(y,UK))) \\
  &\Rightarrow \quad Citizen(x,UK,Birth).
\end{align*}
\end{answer}
\end{shaded}

\item A person born outside the UK, one of whose parents is a UK citizen by birth, is a UK citizen by descent.
\begin{shaded}
\begin{answer}
\begin{align*}
&\forall x \, Person(x) \wedge \neg Born(x,UK) \wedge (\exists y \, Parent(y, x) \wedge Citizen(y,UK,Birth)) \\
& \Rightarrow \quad Citizen(x,UK,Descent).
\end{align*}
\end{answer}
\end{shaded}

\item Politicians can fool some of the people all of the time, and they can fool all of the people some of the time, but they can't fool all of the people all of the time.
\begin{shaded}
\begin{answer}
\begin{align*}
  \forall x Politician(x) & : \\
   &\Rightarrow \quad (\exists y \, \forall t \, Person(y) \wedge Fools(x, y, t)) \wedge (\exists t \, \forall y \, Person(y) \\
  & \Rightarrow \quad Fools(x, y, t)) \wedge \neg (\forall t \, \forall y \, Person(y) \Rightarrow \quad Fools(x, y, t))
\end{align*}
\end{answer}
\end{shaded}


\end{enumerate}
\section{Germans}
 Represent the sentence "All Germans speak the same
languages" in predicate calculus. Use $Speaks(x,l)$, meaning that person $x$ speaks
language $l$, and $German(y)$ , meaning that $y$ is a $German$ person.
\begin{shaded}
\begin{answer}
%\begin{proof}
\begin{equation*}
  \forall x , y , l \quad \left(German(x) \wedge German(y) \wedge Speaks (x,l) \quad \Rightarrow \quad Speaks(y,l) \right)
\end{equation*}
or
\begin{equation*}
  \forall x , y \quad \left(German(x) \wedge German(y)   \quad \Rightarrow \quad \forall l \quad (Speaks (x,l) \quad \Leftrightarrow \quad Speaks(y,l)) \right)
\end{equation*}

%\end{proof}
\end{answer}
\end{shaded}

\section{Jim \& Laura}
What axiom is needed to infer the fact \textit{Female(Laura)} given the facts \textit{Male(Jim)} and \textit{Spouse(Jim, Laura)}?

\begin{shaded}
\begin{answer}
ANSWER HERE.
\end{answer}
\end{shaded}

\section{Describing the predicates}
Write axioms describing the predicates: $GrandChild$, $GreatGrandparent$, $Brother$, $Sister$, $Daughter$, $Son$, $Aunt$, $Uncle$, $BrotherInLaw$, $SisterInLaw$ and $FirstCousin$. Find out the proper definition of math cousin $n$ times removed, and write the definition in first-order logic.
\begin{shaded}
\begin{answer}
$$ $$
\begin{itemize}
  \item $ GrandChild(a,b) \quad \Rightarrow \quad parent(b,x) \wedge parent(x,a) $
  \item $ GreatGrandparent(a,b) \quad \Rightarrow \quad parent(a,x) \wedge GrandChild(b,x) $
  \item The Sibling relationship is added to make the expression of some future relationships simpler. In this situation, Sibling encompasses full, half and step siblings.
      \begin{itemize}
        \item $ Sibling(a,b) \quad \Rightarrow \quad  parent(x,a) \wedge parent(x,b) \wedge not\_equal(a,b) $
        \item $ Sibling(a,b) \quad \Rightarrow \quad Sibling(b,a) $
      \end{itemize}
  \item $Brother(a,b) \quad \Rightarrow \quad Sibling(a,b) \wedge gender(a,\text{`male'})$
  \item $Sister(a,b) \quad \Rightarrow \quad Sibling(a,b) \wedge gender(a,\text{`female'})$
  \item $Son(a,b) \quad \Rightarrow \quad parent(b,a) \wedge gender(a,\text{`male'})$
  \item $ Daughter(a,b) \quad \Rightarrow \quad parent(b,a) \wedge gender(a,\text{`female'})$
  \item $ Uncle(a,b) \quad \Rightarrow \quad parent(x,b) \wedge Sibling(x,a) \wedge gender(a,\text{`male'}) $
  \item $ Aunt(a,b) \quad \Rightarrow \quad parent(x,b) \wedge Sibling(x,a) \wedge gender(a,\text{`female'}) $
  \item married is a primitive relation, meaning it is not defined in terms of any other relations. It is necessary however to note that it is reflexive.
      \begin{itemize}
        \item $married(a,b) \quad \Rightarrow \quad married(b,a)$
      \end{itemize}
  \item $ BrotherInLaw(a,b) \quad \Rightarrow \quad married(b,x) \wedge Sibling(a,x) \wedge gender(a,\text{`male'}) $
  \item $ SisterInLaw(a,b) \quad \Rightarrow \quad married(b,x) \wedge Sibling(a,x) \wedge gender(a,\text{`female'}) $
  \item $ FirstCousin(a,b) \quad \Rightarrow \quad parent(x,a) \wedge parent(y,b) \wedge Sibling(x,y) $
\end{itemize}
\end{answer}
\end{shaded}

\section{Humanoid wolf}
QUESTION HERE.
\begin{enumerate}[label*=\roman*.,ref=\roman*]

\item ITEM
\begin{shaded}
\begin{answer}
ANSWER HERE.
\end{answer}
\end{shaded}

\item ITEM
\begin{shaded}
\begin{answer}
ANSWER HERE.
\end{answer}
\end{shaded}

\item ITEM
\begin{shaded}
\begin{answer}
ANSWER HERE.
\end{answer}
\end{shaded}

\item ITEM
\begin{shaded}
\begin{answer}
ANSWER HERE.
\end{answer}
\end{shaded}

\item ITEM
\begin{shaded}
\begin{answer}
ANSWER HERE.
\end{answer}
\end{shaded}

\item ITEM
\begin{shaded}
\begin{answer}
ANSWER HERE.
\end{answer}
\end{shaded}

\end{enumerate}

\end{document} 