% IN THE NAME OF GOD
% LaTeX Problem Set Template by Sachin Padmanabhan edited by Ali Heydari
% I created this when I was a freshman in CS 103,
% and I continue to use it to this day.
%
% Hope you enjoy!
%
% There may be problems with this template.
% If so, feel free to contact me.
%

\documentclass[a4paper]{article}
\usepackage{amsmath}
\usepackage{amssymb}
\usepackage{amsthm}
\usepackage{amssymb}
\usepackage{mathdots}
\usepackage[pdftex]{graphicx}
\usepackage{fancyhdr}
\usepackage[margin=1in]{geometry}
\usepackage{multicol}
\usepackage{bm}
\usepackage{listings}
\PassOptionsToPackage{usenames,dvipsnames}{color}  %% Allow color names
\usepackage{pdfpages}
\usepackage{algpseudocode}
\usepackage{tikz}
\usepackage{enumitem}
\usepackage[T1]{fontenc}
\usepackage{inconsolata}
\usepackage{framed}
\usepackage{wasysym}
\usepackage[thinlines]{easytable}
\usepackage{hyperref}
\usepackage{dsfont}


\usepackage{wrapfig}
\setlength{\intextsep}{0pt}
\setlength{\columnsep}{0pt}
\usepackage{subcaption}
\usepackage{graphicx}
\graphicspath{ {images/} }

\hypersetup{
    colorlinks=true,
    linkcolor=blue,
    filecolor=magenta,
    urlcolor=blue,
}

\title{
\textsc{Iran university of science and technology} \\ [25pt] % Your university, school and/or department name(s)
Discrete mathematics\\Problem Set \#4 \\
}
\author{Ali Heydari}
\date{\today}

\lhead{Ali Heydari}
\chead{Problem Set \#4}
\rhead{\today}
\lfoot{}
\cfoot{Discrete mathematics --- Spring 2018}
\rfoot{\thepage}

\newcommand{\abs}[1]{\lvert #1 \rvert}
\newcommand{\absfit}[1]{\left\lvert #1 \right\rvert}
\newcommand{\norm}[1]{\left\lVert #1 \right\rVert}
\newcommand{\eval}[3]{\left[#1\right]_{#2}^{#3}}
\renewcommand{\(}{\left(}
\renewcommand{\)}{\right)}
\newcommand{\floor}[1]{\left\lfloor#1\right\rfloor}
\newcommand{\ceil}[1]{\left\lceil#1\right\rceil}
\newcommand{\pd}[1]{\frac{\partial}{\partial #1}}
\newcommand{\inner}[1]{\langle#1\rangle}
\newcommand{\cond}{\bigg|}
\newcommand{\rank}[1]{\mathbf{rank}(#1)}
\newcommand{\range}[1]{\mathbf{range}(#1)}
\newcommand{\nullsp}[1]{\mathbf{null}(#1)}
\newcommand{\repr}[1]{\left\langle#1\right\rangle}

\DeclareMathOperator{\Var}{Var}
\DeclareMathOperator{\tr}{tr}
\DeclareMathOperator{\Tr}{\mathbf{Tr}}
\DeclareMathOperator{\diag}{\mathbf{diag}}
\DeclareMathOperator{\dist}{\mathbf{dist}}
\DeclareMathOperator{\prob}{\mathbf{prob}}
\DeclareMathOperator{\dom}{\mathbf{dom}}
\DeclareMathOperator{\E}{\mathbf{E}}
\DeclareMathOperator{\R}{\mathbb{R}}
\DeclareMathOperator{\var}{\mathbf{var}}
\DeclareMathOperator{\quartile}{\mathbf{quartile}}
\DeclareMathOperator{\conv}{\mathbf{conv}}
\DeclareMathOperator{\VC}{VC}
\DeclareMathOperator*{\argmax}{arg\,max}
\DeclareMathOperator*{\argmin}{arg\,min}
\DeclareMathOperator{\Ber}{Bernoulli}
\DeclareMathOperator{\NP}{\mathbf{NP}}
\DeclareMathOperator{\coNP}{\mathbf{coNP}}
\DeclareMathOperator{\TIME}{\mathsf{TIME}}
\DeclareMathOperator{\polytime}{\mathbf{P}}
\DeclareMathOperator{\PH}{\mathbf{PH}}
\DeclareMathOperator{\SIZE}{\mathbf{SIZE}}
\DeclareMathOperator{\ATIME}{\mathbf{ATIME}}
\DeclareMathOperator{\SPACE}{\mathbf{SPACE}}
\DeclareMathOperator{\ASPACE}{\mathbf{ASPACE}}
\DeclareMathOperator{\NSPACE}{\mathbf{NSPACE}}
\DeclareMathOperator{\Z}{\mathbb{Z}}
\DeclareMathOperator{\N}{\mathbb{N}}
\DeclareMathOperator{\EXP}{\mathbf{EXP}}
\DeclareMathOperator{\NEXP}{\mathbf{NEXP}}
\DeclareMathOperator{\NTIME}{\mathbf{NTIME}}
\DeclareMathOperator{\DTIME}{\mathbf{DTIME}}
\DeclareMathOperator{\poly}{poly}
\DeclareMathOperator{\BPP}{\mathbf{BPP}}
\DeclareMathOperator{\ZPP}{\mathbf{ZPP}}
\DeclareMathOperator{\RP}{\mathbf{RP}}
\DeclareMathOperator{\coRP}{\mathbf{coRP}}
\DeclareMathOperator{\BPL}{\mathbf{BPL}}
\DeclareMathOperator{\IP}{\mathbf{IP}}
\DeclareMathOperator{\PSPACE}{\mathbf{PSPACE}}
\DeclareMathOperator{\NPSPACE}{\mathbf{NPSPACE}}
\DeclareMathOperator{\SAT}{\mathsf{SAT}}
\DeclareMathOperator{\NL}{\mathbf{NL}}
\DeclareMathOperator{\PCP}{\mathbf{PCP}}
\DeclareMathOperator{\PP}{\mathbf{PP}}
\DeclareMathOperator{\cost}{cost}
\let\Pr\relax
\DeclareMathOperator*{\Pr}{\mathbf{Pr}}

\definecolor{shadecolor}{gray}{0.95}

\theoremstyle{plain}
\newtheorem*{lem}{Lemma}

\theoremstyle{plain}
\newtheorem*{claim}{Claim}

\theoremstyle{definition}
\newtheorem*{answer}{Answer}

\newtheorem{theorem}{Theorem}[section]
\newtheorem*{thm}{Theorem}
\newtheorem{corollary}{Corollary}[theorem]
\newtheorem{lemma}[theorem]{Lemma}

\renewcommand{\headrulewidth}{0.4pt}
\renewcommand{\footrulewidth}{0.4pt}

\setlength{\parindent}{0pt}

\pagestyle{fancy}

\renewcommand{\thefootnote}{\fnsymbol{footnote}}

\begin{document}

\maketitle

\section{Prove the sequence}
Consider the sequence defined recursively by:
\begin{align*}
   & a_0 = 1 \\
   & a_n = a_{n-1} + a_{n-2} + \dots + a_0 +1
\end{align*}
Prove that $a_n=2^n$ by strong induction.

\begin{shaded}
\begin{answer}
\begin{proof}
PROOF HERE.
\end{proof}
\end{answer}
\end{shaded}

\section{Fill the board using  L shaped tiles}
Given a $n$ by $n$ board where $n$ is of form $2k$ where $k \geq 1$ (Basically n is a power of 2 with minimum value as 2). The board has one missing cell (of size $1 \times 1$). Prove that the board can be filled using $L$ shaped tiles.

 A $L$ shaped tile is $2 \times 2$ square with one cell of size $1\times 1$ missing.

\includegraphics[scale = 0.2]{LShapes}

\begin{shaded}
\begin{answer}
\begin{proof}
PROOF HERE.
\end{proof}
\end{answer}
\end{shaded}

\section{Prime number and positive number}
prove that every positive integer $n$, $n \geq 2$, can be expressed as the product of one or more prime numbers.
\begin{shaded}
\begin{answer}
\begin{proof}
PROOF HERE.
\end{proof}
\end{answer}
\end{shaded}

\section{The game}
In a game , there are two players and two piles of matches. At each turn, a player removes some (non-zero) number of matches from one of the piles. The player who removes the last match wins.

Prove that if the two piles contain the same number of matches at the start of the game, then the second player can always win.

\begin{shaded}
\begin{answer}
\begin{proof}
PROOF HERE.
\end{proof}
\end{answer}
\end{shaded}

\section{ATM Machine}
Suppose an ATM machine has only $\$2$ and $\$5$ bills.
Claim: The ATM can generate any output amount $n \geq 4$.
Prove the claim.
\begin{shaded}
\begin{answer}
\begin{proof}
PROOF HERE.
\end{proof}
\end{answer}
\end{shaded}

\section{Colored computers}
We have $n$ computers ($n \geq 4$), each pair of them are connected by a wire, each wire is colored in blue, red, yellow or green. At least there is a wire colored in each color.

Prove that we can choose a number of computers so that the wires connecting computers, include exactly 3 different colors.

\begin{shaded}
\begin{answer}
\begin{proof}
PROOF HERE.
\end{proof}
\end{answer}
\end{shaded}


\end{document} 