% IN THE NAME OF GOD
% LaTeX Problem Set Template by Sachin Padmanabhan edited by Ali Heydari
% I created this when I was a freshman in CS 103,
% and I continue to use it to this day.
%
% Hope you enjoy!
%
% There may be problems with this template.
% If so, feel free to contact me.
%

\documentclass[a4paper]{article}
\usepackage{amsmath}
\usepackage{amssymb}
\usepackage{amsthm}
\usepackage{amssymb}
\usepackage{mathdots}
\usepackage[pdftex]{graphicx}
\usepackage{fancyhdr}
\usepackage[margin=1in]{geometry}
\usepackage{multicol}
\usepackage{bm}
\usepackage{listings}
\PassOptionsToPackage{usenames,dvipsnames}{color}  %% Allow color names
\usepackage{pdfpages}
\usepackage{algpseudocode}
\usepackage{tikz}
\usepackage{enumitem}
\usepackage[T1]{fontenc}
\usepackage{inconsolata}
\usepackage{framed}
\usepackage{wasysym}
\usepackage[thinlines]{easytable}
\usepackage{hyperref}
\usepackage{dsfont}


\usepackage{wrapfig}
\setlength{\intextsep}{0pt}
\setlength{\columnsep}{0pt}
\usepackage{subcaption}
\usepackage{graphicx}
\graphicspath{ {images/} }

\hypersetup{
    colorlinks=true,
    linkcolor=blue,
    filecolor=magenta,
    urlcolor=blue,
}

\title{
\textsc{
\center
\includegraphics[scale=0.5]{IUST_logo_color} \\
Iran university of science and technology} \\ [5pt] % Your university, school and/or department name(s)
Department of Computer Engineering \\[8pt]
Discrete Mathematics\\Problem Set \#5 \\
}
\author{Ali Heydari}
\date{\today}

\lhead{Ali Heydari}
\chead{Problem Set \#5}
\rhead{\today}
\lfoot{}
\cfoot{Discrete mathematics --- Spring 2018}
\rfoot{\thepage}

\newcommand{\abs}[1]{\lvert #1 \rvert}
\newcommand{\absfit}[1]{\left\lvert #1 \right\rvert}
\newcommand{\norm}[1]{\left\lVert #1 \right\rVert}
\newcommand{\eval}[3]{\left[#1\right]_{#2}^{#3}}
\renewcommand{\(}{\left(}
\renewcommand{\)}{\right)}
\newcommand{\floor}[1]{\left\lfloor#1\right\rfloor}
\newcommand{\ceil}[1]{\left\lceil#1\right\rceil}
\newcommand{\pd}[1]{\frac{\partial}{\partial #1}}
\newcommand{\inner}[1]{\langle#1\rangle}
\newcommand{\cond}{\bigg|}
\newcommand{\rank}[1]{\mathbf{rank}(#1)}
\newcommand{\range}[1]{\mathbf{range}(#1)}
\newcommand{\nullsp}[1]{\mathbf{null}(#1)}
\newcommand{\repr}[1]{\left\langle#1\right\rangle}

\DeclareMathOperator{\Var}{Var}
\DeclareMathOperator{\tr}{tr}
\DeclareMathOperator{\Tr}{\mathbf{Tr}}
\DeclareMathOperator{\diag}{\mathbf{diag}}
\DeclareMathOperator{\dist}{\mathbf{dist}}
\DeclareMathOperator{\prob}{\mathbf{prob}}
\DeclareMathOperator{\dom}{\mathbf{dom}}
\DeclareMathOperator{\E}{\mathbf{E}}
\DeclareMathOperator{\R}{\mathbb{R}}
\DeclareMathOperator{\var}{\mathbf{var}}
\DeclareMathOperator{\quartile}{\mathbf{quartile}}
\DeclareMathOperator{\conv}{\mathbf{conv}}
\DeclareMathOperator{\VC}{VC}
\DeclareMathOperator*{\argmax}{arg\,max}
\DeclareMathOperator*{\argmin}{arg\,min}
\DeclareMathOperator{\Ber}{Bernoulli}
\DeclareMathOperator{\NP}{\mathbf{NP}}
\DeclareMathOperator{\coNP}{\mathbf{coNP}}
\DeclareMathOperator{\TIME}{\mathsf{TIME}}
\DeclareMathOperator{\polytime}{\mathbf{P}}
\DeclareMathOperator{\PH}{\mathbf{PH}}
\DeclareMathOperator{\SIZE}{\mathbf{SIZE}}
\DeclareMathOperator{\ATIME}{\mathbf{ATIME}}
\DeclareMathOperator{\SPACE}{\mathbf{SPACE}}
\DeclareMathOperator{\ASPACE}{\mathbf{ASPACE}}
\DeclareMathOperator{\NSPACE}{\mathbf{NSPACE}}
\DeclareMathOperator{\Z}{\mathbb{Z}}
\DeclareMathOperator{\N}{\mathbb{N}}
\DeclareMathOperator{\EXP}{\mathbf{EXP}}
\DeclareMathOperator{\NEXP}{\mathbf{NEXP}}
\DeclareMathOperator{\NTIME}{\mathbf{NTIME}}
\DeclareMathOperator{\DTIME}{\mathbf{DTIME}}
\DeclareMathOperator{\poly}{poly}
\DeclareMathOperator{\BPP}{\mathbf{BPP}}
\DeclareMathOperator{\ZPP}{\mathbf{ZPP}}
\DeclareMathOperator{\RP}{\mathbf{RP}}
\DeclareMathOperator{\coRP}{\mathbf{coRP}}
\DeclareMathOperator{\BPL}{\mathbf{BPL}}
\DeclareMathOperator{\IP}{\mathbf{IP}}
\DeclareMathOperator{\PSPACE}{\mathbf{PSPACE}}
\DeclareMathOperator{\NPSPACE}{\mathbf{NPSPACE}}
\DeclareMathOperator{\SAT}{\mathsf{SAT}}
\DeclareMathOperator{\NL}{\mathbf{NL}}
\DeclareMathOperator{\PCP}{\mathbf{PCP}}
\DeclareMathOperator{\PP}{\mathbf{PP}}
\DeclareMathOperator{\cost}{cost}
\let\Pr\relax
\DeclareMathOperator*{\Pr}{\mathbf{Pr}}

\definecolor{shadecolor}{gray}{0.95}

\theoremstyle{plain}
\newtheorem*{lem}{Lemma}

\theoremstyle{plain}
\newtheorem*{claim}{Claim}

\theoremstyle{definition}
\newtheorem*{answer}{Answer}

\newtheorem{theorem}{Theorem}[section]
\newtheorem*{thm}{Theorem}
\newtheorem{corollary}{Corollary}[theorem]
\newtheorem{lemma}[theorem]{Lemma}

\renewcommand{\headrulewidth}{0.4pt}
\renewcommand{\footrulewidth}{0.4pt}

\setlength{\parindent}{0pt}

\pagestyle{fancy}

\renewcommand{\thefootnote}{\fnsymbol{footnote}}

\begin{document}
\maketitle
\tableofcontents
\pagebreak

\section{Connected graph}
Prove the theorem blow.
\begin{theorem}
A graph is connected if and only if for every partition of it's vertices into two nonempty sets, there
is an edge with endpoints in both sets.
\end{theorem}
\begin{shaded}
\begin{answer}
\begin{proof}
$(\Rightarrow)$ Let $G$ is connected. Take an arbitrary partition $V (G) = X \cup Y of V (G)$ into two non-empty sets. We should show that $G$ has an edge with one endpoint in $X$ and the other in $Y$. Select vertices $x$ in $X$, and $y$ in $Y$ . (Possible because $X$ and $Y$ are non-empty.) Because $G$ is connected, there has to be a path in $G$ that joins $x$ to $y$. Denote this path as follows.
$$x = x_0, x_1, x_2, x_3, \dots , x_n = y$$
The first vertex $x_0$ of this path is in $X$, and the last vertex $x_n$ is in $Y$ . Any one of the others is either in $X$ or $Y$ . Suppose $i$ be the smallest index for which $x_i \in Y$ . (Such an $i$ exists, because $x_n \in Y$ , so $i$ is at most $n$.) Now we have $x_{i-1} \in X$ and $x_i \in Y$ , so $x_{i-1}x_i$
is an edge of $G$ with one endpoint in $X$ and the other in $Y$ .

$(\Leftarrow)$ Let that for any partition $V (G) = X \cup Y of V (G)$ into two non-empty sets, $G$ has an edge with one endpoint in $X$ and the other in $Y$ . We need to show $G$ is connected. For the sake of contradiction, suppose $G$ is not connected. Let $C$ be one of its components. Now we have a partition
$$V (G) = V (C) \in (V (G) − V (C))$$
 of $V (G)$ into two non-empty sets. By assumption, $G$ has an edge with endpoints in each set in this partition. That is to say $G$ has an edge with one endpoint in one of its components and the other endpoint in another component. Contradiction!
 \end{proof}
\end{answer}
\end{shaded}

\section{$n$-vertex graph}
Prove the theorem blow.
\begin{theorem}
Every $n$-vertex graph with at least $n$ edges contains a cycle.
\end{theorem}
\begin{shaded}
\begin{answer}
\begin{proof}
\textbf{Assume:} $G$ contains no cycles. Then every connected component of $G$ is a tree.

 \textbf{Claim:} The number of edges in a tree on $n$ vertices is $n-1$.

Proof is by induction. The claim is obvious for $n=1$. Assume that it holds for trees on $n$ vertices. Take a tree on $n+1$ vertices. It's an easy exercise (look at a longest path in $G$) to show that a tree has at least one terminal vertex  (i.e. with degree 1). Removing this terminal vertex along with its edge, we get a tree on $n$ vertices, and induction takes us home. Hence the number of edges in a graph without cycles is $n-k$, where $k$ is the number of connected components.
\end{proof}
\end{answer}
\end{shaded}




\section{$k\_cube$ graph}
what is $k\_cube(Qk)$ graph?

Count number of it's Vertices and Edges and prove that it is bipartite.
\begin{shaded}
\begin{answer}
In graph theory, the hypercube graph $Q_n$ is the graph formed from the vertices and edges of an $n$-dimensional hypercube. For instance, the cubical graph $Q_3$ is the graph formed by the 8 vertices and 12 edges of a three-dimensional cube. $Q_n$ has $2n$ vertices, $2^{n-1}n$ edges, and is a regular graph with $n$ edges touching each vertex.\footnote{ \href{https://en.wikipedia.org/wiki/Hypercube_graph}{From Wikipedia, the free encyclopedia}}
\begin{proof}
There are 2 possibilities for each of the $k$ coordinates of a vertex, so there are $2k$ vertices in total. Each vertex has $k$ neighbours (why?) so the sum of the degrees of all the vertices is $k2^k$. But this counts each edge twice, so...

To show that the graph is bipartite, consider the coordinate sum of a vertex. If this sum is even, what can you say about the sums for neighbouring vertices? What if the sum is odd?
\end{proof}
\end{answer}
\end{shaded}

\section{Diameter of graph}
diameter of $G$ is length of the longest path between two vertices in it.
Show that if $G$'s diameter is greater than 3 its complement's diameter would be less than 3.
\begin{shaded}
\begin{answer}
\begin{proof}
Denote distance in $G$ by $d$ and distance in $G$' by $d$'. Define $u,v$ as you did, so that $d(u,v) \geq 4$. We wish to show for any vertices $x,y$, that $d'(x,y) \leq 2$.

Among $x,y,u,v$ there are at most four distinct vertices. In particular, since $d(u,v) \geq 4$, $u$ and $v$ cannot be connected by a path among $x,y,u,v$. So it is possible to split $x,y,u,v$ into two components which are not connected in $G$, one containing u and one containing $v$. WLOG we may assume the components are $\{u,x\},\{v,y\}$ or $\{u,x,y\},\{v\}$.

In the first case, $x$ and $y$ have no edge in $G$, so they have an edge in $G$' and $d'(x,y)=1$.

In the second case, $x$ and $y$ are both connected to $v$ in $G$', so $d'(x,y) \leq d'(x,v) + d'(y,v)=2$.

Thus $d'(x,y) \leq 2$ for any $x,y$, proving that $\text{diam}G' \leq 2$.
\end{proof}
\end{answer}
\end{shaded}


\section{Strong orientation of a graph}
A strong orientation of a graph that has an odd cycle also has an odd $(directed)$ cycle. Suppose that $D$ is a strong orientation of a graph $G$ that has an odd cycle $v1, \dots , v2k+1$. Since $D$ is strongly connected, for each $i$ there is a $vi, vi+1-path$ in $D$. If for some $i$ every such path has even length, then the edge between $vi$ and $vi+1$ points from $vi+1$ to $vi$, since the other orientation would be a $vi, vi+1-path$ of length 1 $(odd)$. In this case, we have an odd cycle through $vi$ and $vi+1$. Otherwise, we have a path of odd length from each $vi$ to $vi+1$. Combining these gives a closed trail of odd length In a digraph as well as in a graph (by the same proof), a closed odd trail contains the edges of an odd cycle
\begin{shaded}
\begin{answer}
ANSWER HERE.
\end{answer}
\end{shaded}



\section{About Tree}
For a tree $T$ with vertex degrees in $\{1, k\}$, the possible values of $n(T)$ are the positive integers that are 2 more than a multiple of $k − 1$.
\begin{shaded}
\begin{answer}
(induction on $m$, the number of vertices of degree $k$). We proof that if $T$ has $m$ vertices of degree $k$, then $n(T) = m(k -1) + 2$ If $m = 0$, then the tree must have two vertices.

For the induction step, suppose that $m > 0$. For a tree $T$ with $m$ vertices of degree $k$ and the rest of degree 1, let $T'$  be the tree obtained by deleting all the leaves. The tree $T'$  is a tree whose vertices all have degree $k$ in $T$ . Let $x$ be a leaf of $T'$. In $T$ , $x$ is adjacent to one non­leaf and to $k -1$ leaves. Deleting the leaf neighbors of $x$ leaves a tree $T''$  with $m - 1$ vertices of degree k and the rest of degree 1. By the induction hypothesis, $n(T'' ) = (m -�1)(k �-1) +2$. Since we deleted $k �-1 $ vertices from $T$ to obtain $T''$, we obtain $n(T) = m(k -�1) + 2$. This completes the induction step.

To prove inductively that all such values arise as the number of vertices in such a tree, we start with $K_2$ and iteratively expand a leaf into a vertex of degree $k$ to add $k - 1$ vertices.
\end{answer}
\end{shaded}



\section{More about Tree}
Prove the theorems blow.
\begin{enumerate}[label=\Roman*]
  \item {
  \begin{theorem}
A tree has exactly one center or has two adjacent centers.
\end{theorem}
\begin{shaded}
\begin{answer}
\begin{proof}
We will use one observation that the maximum distance $max d(v,w)$ from a given vertex $v$ to any other vertex $w$ occurs only when $w$ is pendant vertex.

Now, let $T$ is a tree with $n$ vertices $(n \geq 2)$

$\Rightarrow$  $T$ must have at least two pendant vertices.

delete all pendant vertices from $T$, then resulting graph $T’$ is still a tree.

$\Rightarrow$ eccentricity $E(v)$ in $T’$ is just one less than $E(v)$ in $T$ $\forall v$ in $T’$

again delete pendant vertices from $T’$ so that resulting $T''$ is still a tree with same centers.

Note that all vertices that $T$ had as centers will still remain centers in $T’ \rightarrow T'' \rightarrow T''' \rightarrow \dots $

continue this process untill remaining tree has either one vertex or one edge.

So at the end, if one vertex is there this implies tree $T$ has one center.

If one edge is there then tree $T$ has two centers.
\end{proof}
\end{answer}
\end{shaded}
}

  \item {
  \begin{theorem}
A tree has exactly one center if and only if its diameter is twice its radius.
\end{theorem}
\begin{shaded}
\begin{answer}
\begin{proof}
  above observes that the center or pair of centers is the middle of a longest path. The diameter of a tree is the length of its longest path. The radius is the eccentricity of any center. If the diameter is even, then there is one center, and its eccentricity is half the length of the longest path. If the diameter is odd, say $2k �- 1$, then there are two centers, and the eccentricity of each is $k$, which exceeds $(2k �- 1/=2)$.
\end{proof}
\end{answer}
\end{shaded}
}
\end{enumerate}

\section{Certain bridge club}
A certain bridge club has a special rule to the effect that four members may play together only if no two of them have previously partnered one another. At one meeting fourteen members, each of whom has previously partnered five others, turn up. Three games are played, and then proceedings come to a halt because of the club rule. Just as the members are preparing to leave, a new member, unknown to any of them, arrives. Show that at least one more game can now be played.
\begin{shaded}
\begin{answer}
This problem relies on a theorem by the Hungarian mathematician Turan which says that if $G$ contains no $K_3$,  then $|EG| \leq |EB_v|$, where $v = |VG|$  and $B_v$ is a bipartite graph in which each part has as close to $\frac{v}{2}$ vertices as possible. It turns out that  $ |EB_v| = \, ^{v-k} \!  C_2 + \, ^{k+1} \!  C_2 $, where $ k = \lfloor \frac{n}{2} \rfloor $  (the integral part of $ \frac{1}{2} n $).

In the problem let $G$ be the graph whose vertices are the people in the club and whose edges represent people who have not yet partnered each other. If this graph has no triangle, then when the new member arrives no game is possible.

\end{answer}
\end{shaded}

\end{document} 